\documentclass[12pt,a4paper]{article}
\usepackage{amsmath}
\begin{document}
	This is some text
	%Comment
	\begin{align}\label{equ1(good)}%Comment
	(5+5*2+(5-2))%Comment
	\end{align}
	This is some text with 2\%.
	\begin{align*}
	&\left(5+5*2+\left(5-2\%\right)\right)\\
	&\left(5+5*2+\left(5-2\%\right)\right)%Already have autosize param
	\end{align*}
	\begin{equation}\label{equ2(good)}
	(5+5*2+(5-2))%Comment (5+5)
	%Comment(5+5)
	\end{equation}
	\begin{equation*}
	(5+5*2+(5-2))
	\end{equation*}
	\begin{equation*}
	\frac{7}{2}\left(\frac{25}{80}-\frac{a(2+b)}{384}\right) = (5+2)\left(\frac{5(3+2)}{(8+2)(5+3)}-\frac{a(2+b)}{8(5+3)(2+4)}\right)
	\end{equation*}
	\((5+5*2+(5-2))\)
	\[(5+5*2+(5-2))\]
	The following is an inline equation $5(5+x)$ as you can see.
	\begin{align*}
	&\left(5+5*2+\left(5-2^2\right)\right)\\
	&\left(5+5*2+\left(5-2_2\right)\right)
	\end{align*}
\end{document}
